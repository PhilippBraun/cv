\documentclass[a4paper,10pt]{article}

%A Few Useful Packages
\usepackage{marvosym}
\usepackage{fontspec} 					%for loading fonts
\usepackage{xunicode,xltxtra,url,parskip} 	%other packages for formatting
\RequirePackage{color,graphicx}
\usepackage[usenames,dvipsnames]{xcolor}
%\usepackage[big]{layaureo} 				%better formatting of the A4 page
% an alternative to Layaureo can be ** \usepackage{fullpage} **
\usepackage{supertabular} 				%for Grades
\usepackage{titlesec}					%custom \section
\usepackage{tabularx}

\usepackage[top=1cm, bottom=1cm, left=1cm, right=1cm]{geometry}

%Setup hyperref package, and colours for links
\usepackage{hyperref}
\definecolor{linkcolour}{rgb}{0,0.2,0.6}
\hypersetup{colorlinks,breaklinks,urlcolor=linkcolour, linkcolor=linkcolour}

\newcolumntype{L}{>{\raggedright\arraybackslash}X}

%FONTS
\defaultfontfeatures{Mapping=tex-text}
\setmainfont[SmallCapsFont = Fontin SmallCaps]{Fontin}

%CV Sections inspired by: 

\titleformat{\section}{\Large\scshape\raggedright}{}{0em}{}[\titlerule]
\titlespacing{\section}{0pt}{3pt}{3pt}
%Tweak a bit the top margin
%\addtolength{\voffset}{-1.3cm}

%Italian hyphenation for the word: ''corporations''
\hyphenation{im-pre-se}

%-------------WATERMARK TEST [**not part of a CV**]---------------
\usepackage[absolute]{textpos}

\setlength{\TPHorizModule}{30mm}
\setlength{\TPVertModule}{\TPHorizModule}
\textblockorigin{2mm}{0.65\paperheight}
\setlength{\parindent}{0pt}

%--------------------BEGIN DOCUMENT----------------------
\begin{document}


\pagestyle{empty} % non-numbered pages

\font\fb=''[cmr10]'' %for use with \LaTeX command

%--------------------TITLE-------------
\par{\centering
		{\Large Curriculum Vitae}\vspace{0.3cm}\\
		{\Huge Philipp \textsc{Braun}
	}\bigskip\par}

%--------------------SECTIONS-----------------------------------
%Section: Personal Data
%\section{Personal Data}

\begin{tabularx}{19cm}{LLLL}
    \textbf{Date of Birth:} & 12 November 1985 & \textbf{Address:}& \textbf{Phone:} +49 176 63651644\\
    \textbf{Place of Birth:} & Neumarkt i.d.Opf  & Marienroggenweg 51 & \textbf{Email:}  \href{mailto:braunp@in.tum.de}{braunp@in.tum.de}\\
    \textbf{Nationality:} & German (EU citizenship) & 18147 Rostock& 
\end{tabularx}

%Section: Education
\section{Education}
\begin{tabularx}{19cm}{rX}	

\textsc{Oct 2011}& \textbf{M.Sc., Applied \& Engineering Physics} \hfill{Technische Universität München}\\
\textsc{- Sep 2013} & \textbf{Laurea Magistrale, Materials Science} \hfill{University of Turin}\\
 (expected)&  \textbf{Master 2, Chemistry} \hfill{University of Montpellier 2}\\
 &  \textsc{current Gpa}: 1.7 (German grades range from 1.0 \small{(excellent)} to 5.0 \small{(fail)})\\& \small{Erasmus Mundus Master programme Materials Science Exploring Large Scale Facilities (MaMaSELF) leading to a
triple degree}\\\\

\textsc{Oct 2009}& \textbf{B.Sc., Informatics} \hfill{Technische Universität München}\\
\textsc{- Sep 2012} & Application Area: Physics\\
 & \textsc{Gpa}: 1.8, \small{Thesis title: ”Probe position refinement in X-ray ptychography"}\\
& Advisor: Prof. Thomas Huckle (TUM Informatics), Pierre Thibault, PhD. (TUM Physics)
\\ \\

\textsc{Oct 2008}& \textbf{B.Sc.,  Physics} \hfill\normalsize{Technische Universität München}\\
 \textsc{- Sep 2011}& Specialization Area: Condensed Matter Physics \\
& \textsc{Gpa}: 2.2, \small{Thesis title: ”Development and analysis of a momentum spectrometer for charged particles emitted after the photo double ionization of ethyne and ethylene"}\\
& Advisor: Dr. Thorsten Weber (LBNL),  Prof. Peter Müller-Buschbaum (TUM) \\\\


\textsc{Apr 2006}& \textbf{Associate Engineer, Computer Systems and Automation} \hfill  {Siemens Technik Akademie}\\
\textsc{- Mar 2008}& \textsc{Gpa}: 1.0  \hfill Erlangen, Germany\\
 &\\ 

\textsc{May} 2005& \textbf{Abitur},\hfill  Melanchthon Gymnasium \\
&  \textsc{Gpa}: 1.4 \hfill Nuremberg, Germany

\end{tabularx}

%Section: Work Experience at the top
\section{Work Experience}
\begin{tabularx}{19cm}{rX}

 \textsc{Feb 2013} & Research Intern \hfill cSAXS beamline, \textsc{Paul Scherrer Intitute}\\
\textsc{- Sep 2013} &\emph{Coherent Diffractive Imaging} \hfill Villigen, Switzerland\\
(expected)&\footnotesize{Masters Student. Mixed State Reconstruction with X-Ray Ptychography, Advisors: Dr. Andreas Menzel, Dr. Pierre Thibault}\vspace{2mm}\\ 

 \textsc{Jul 2012} & Research Intern \hfill Coherent Imaging Division, \textsc{CFEL, DESY}\\
\textsc{- Sep 2012} &\emph{Coherent Imaging with FELs} \hfill Hamburg, Germany\\
&\footnotesize{Participant in the DESY Summer Programme. 3D phase retrieval algorithms. Binary classification of single-particle diffraction patterns from FELs. Advisors: Dr. Anton Barty, Prof. Henry Chapman}\vspace{2mm}\\ 

 \textsc{Jul 2011} & Research Intern \hfill AMOS Group, \textsc{Lawrence Berkeley National Laboratory}\\
\textsc{- Sep 2011} &\emph{Atomic and Molecular Physics} \hfill Berkeley, U.S.\\
&\footnotesize{Participant in the DAAD RISE programme. Development, simulation and resolution analysis of a 3D momentum spectrometer for charged particles emitted
after the photo double ionization of ethyne and ethylene. Help with setup of the experiment at beamline 10.0.1 of
the Advanced Light Source. Advisor: Dr. Thorsten Weber}\vspace{2mm}\\ 

 \textsc{Oct 2008} & Software Engineer\hfill \textsc{Interasco} GmbH
 \\\textsc{- Nov 2009}&\emph{IT Solutions and Services} \hfill Munich, Germany\\&\footnotesize{Helped defining the architecture for a newly started long term project. Developed an easy to use multithreading framework. Coached the team in using the framework and provided thorough documentation.}\vspace{2mm}\\ 

 \textsc{Apr 2008} & Junior Developer \hfill \textsc{Swinton Collonade Ltd.} \\
\textsc{- Oct 2008}&\emph{Insurance and Financial Services} \hfill  Manchester, U.K.\\
&\footnotesize{Developed enterprise scale object oriented web applications. Developed website frontend with javascript, DHTML, webservices. Monitored website performance and troubleshot defects out of regular working hours.}\vspace{2mm}\\ 

 \textsc{Oct 2007} & Software Developer Intern \hfill \textsc{Siemens Standard Drives}
\\\textsc{- Mar 2008}&\emph{Manufacturing \& Automation} \hfill  Congleton, U.K.
\\&\footnotesize{Developed an application to streamline tracking of defective circuit boards and increase item throughput in the adjoining factory. Developed an application to automate a small assembly line.}\vspace{2mm}\\ 

\end{tabularx}

\section{Computer Skills}
\begin{tabular}{rl}
 Programming:& \textsc{web pages} (Expert),\textsc{Windows applications} (Expert), Qt (Basic)\\
	      & \textsc{Database Design} (Expert)\\
 Programming Languages:& \textsc{C\#} (Expert), \textsc{Visual Basic} (Expert), \textsc{C/C++} (Expert)\\
			& \textsc{SML} (Intermediate), \textsc{SQL} (Intermediate), \textsc{Perl} (Basic),\\
			&\textsc{HTML} (Intermediate), Python (Intermediate)\\
 Computer Algebra Programs: & \textsc{Mathematica} (Intermediate)\\
 Operating Systems:& \textsc{Windows} (Expert), \textsc{Linux} (Intermediate)\\
 Office:& {\fb \LaTeX}\setmainfont[SmallCapsFont=Fontin SmallCaps]{Fontin-Regular}, Office-Suite
\end{tabular}

\section{Lab experience}
\begin{tabular}{r|p{11cm}}
 \textsc{Basic lab course 1} & - Oscillations and chaos - Capillary viscosimeter\\
			     & - Determination of molar mass - Constitutive equation of real gases\\
			     & - Determination of sonic velocity - Dissociation and freezing point depression of $KNO_3$\\
  \textsc{Basic lab course 2} & - Determination of electron charge - Bridge circuit\\
			     & - Creation of ultra high vacuum, measurement of vacuum pressure\\
			     & - Measurement of transmission curves on the oscilloscope\\
			      &- Characteristic lines of transistors\\
			     & - fuel cell characteristics\\
  \textsc{Basic lab course 3} & - measurement and handling of radioactive materials\\
			     & - measurement of electromagnetic fields\\
			     & - diffraction and refraction of light\\
			     & - X-rays: characteristic and continuous spectrum, generation\\
			     & - geometrical optics: lenses, lense systems, principal plane, autocollimation, focal distance\\
			     & - Franck-Hertz-Experiment\\
  \textsc{Advanced lab course} & - x-ray fluorescence spectrometry\\
			     & - atomic force microscopy\\
			     & - molecular motors. fluorescence microscopy\\
			     & - plasma interferometry, He-Ne lasers, laser resonator, Fabry-Perot-interferometer\\
			     & - organic photovoltaic cells\\
			     & - surface plasmons\\
			     & - Mössbauer Effect\\
			     & - Lasers and nonlinear optics\\
			     & - Fourier transform holography\\
  \textsc{Bachelor Thesis} & - synchrotron radiation\\	
			     & - 3rd generation synchrotrons\\			
\end{tabular}

\section{Military Service}
\begin{tabular}{r|p{11cm}}
 \textsc{Jul 2005}& Private First class at \textsc{German Airforce}, Landsberg a. Lech \\\textsc{- Mar 2006}&\emph{Air raid defences}\\
\end{tabular}
%Section: Scholarships and additional info
%\section{Scholarships and Certificates}
%\begin{tabular}{rl}
% \textsc{Sept.} 2006 & Scholarship for graduate students with an outstanding curriculum \footnotesize(\EURcr 30,000)\normalsize\\
%\textsc{June} 2006 & {\textsc{Gmat}\textregistered}\setmainfont[SmallCapsFont=Fontin SmallCaps]{Fontin-Regular}: 730 (\textsc{q:50;v:39}) 96\textsuperscript{th} percentile; \textsc{awa}: 6.0/6.0 (89\textsuperscript{th} percentile)
%\end{tabular}

%Section: Languages
\section{Languages}
\begin{tabular}{rl}
\textsc{English:}&Fluent\\
\textsc{Spanish:}& C1 Level\\
\textsc{Italian:}&  A2 Level\\
\textsc{French:}& A1 Level\\
\end{tabular}

\section{Interests and Activities}
Technology, Philosophy of Science, University Choir, Programming, Travelling\\

%\hypertarget{gmat}{\textsc{Gmat}\setmainfont{LMRoman10 Regular}\textregistered\setmainfont[SmallCapsFont=Fontin-SmallCaps]{Fontin-Regular}}

%\XeTeXpdffile ''GMAT.pdf'' page 1 scaled 800

\end{document}
